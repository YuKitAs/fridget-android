\subsection{Ablauf der Methode createCoolNote()}

Die Methode createCoolNote() erstellt eine Cool Note und platziert sie auf der WG-Pinnwand. Sie wird in diesem Fall in der Activity CreateTextCoolNoteActivity ausgeführt. Als Erstes wird das zugehörige Viewmodel zur Activity, nämlich das CreateCoolNoteViewModel, erstellt bzw. gegettet(?). Darauffolgend wird die Methode createCoolNote(), die in der View-Model-Klasse steht, aufgerufen. In dieser Methode wird der zugehörige Service des Clients, CoolNoteService, erstellt, der dann die Verknüpfung zur Server-Seite herstellt. Die Methode createCoolNote() des Services wird aufgerufen und als Parameter wird eine Instanz des Objektes CoolNote übergeben, welches die vom Benutzer eingetippten Atrribute enthält (Titel, Text-Inhalt, ggfs. Wichtigkeit, Tags, etc.). Der Service schickt ein HTTP POST Request mithilfe des REST-Clients(?) an den Server bzw. an den zugehörigen CoolNoteController. Im CoolNoteController wird die Methode saveCoolNote() aufgerufen, die (irgendwas mit JSON)...? Im CoolNoteService des Servers wird ebenfalls eine Methode saveCoolNote aufgerufen. Das CoolNoteRepository stellt eine Brücke zwischen der Daten und der App dar und speichert nun also die Cool-Note-Daten in der Datenbank ab. Auf dem Rückweg zur Activity wird schlussendlich die erstellte Cool Note übergeben und eine Response gesendet.