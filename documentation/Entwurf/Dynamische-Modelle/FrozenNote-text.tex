\subsection{Ablauf der Methode editFrozenNote()}

Die Methode editFrozenNote() aktualisiert den Inhalt und den Titel einer Frozen Note. Sie wird in der Activity EditFrozenNoteActivity ausgeführt. Als Erstes wird das zugehörige Viewmodel zur Activity, nämlich das FullFrozenNoteViewModel, verwendet. Darauffolgend wird die Methode editFrozenNote(), die in der View-Model-Klasse steht, aufgerufen. In dieser Methode wird der zugehörige Service des Clients, FrozenNoteService, erstellt, der dann die Verknüpfung zur Server-Seite herstellt. Die Methode updateFrozenNote() des Services wird aufgerufen und als Parameter wird die bereits vorhandene Instanz des Objektes FrozenNote übergeben, welches die vom Benutzer eingetippten Attribute enthält (Titel, Text-Inhalt). Es wird ein HTTP PUT Request mithilfe an den Server gesendet bzw. an den zugehörigen FrozenNoteController. Im FrozenNoteController wird die Methode updateFrozenNote() aufgerufen. Im FrozenNoteService des Servers wird ebenfalls eine Methode updateFrozenNote() aufgerufen. Das FrozenNoteRepository ist verantwortlich für die Datenpersistenz und speichert nun also die neuen Frozen-Note-Daten in der Datenbank ab. Auf dem Rückweg zur Activity wird schlussendlich die editierte Frozen Note übergeben und eine Response gesendet.