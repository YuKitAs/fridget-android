\subsection{Ablauf der Methode checkGoogleLoginData()}

Die Methode checkGoogleLoginData() prüft die eingegebenen Google-Login-Daten des Users. Sie wird in der Activity LoginActivity ausgeführt. Als Erstes wird das zugehörige Viewmodel zur Activity, nämlich das LoginViewModel, erstellt bzw. gegettet(?). Darauffolgend wird die Methode checkGoogleLoginData(), die in der View-Model-Klasse steht, aufgerufen. In dieser Methode wird der zugehörige Service des Clients, UserService, erstellt, der dann die Verknüpfung zur Server-Seite herstellt. Die Methode getGoogleIdToken des Services wird aufgerufen und als Parameter werden die vom User eingegebenen Login-Daten (E-Mail, Passwort) übergeben. Der Service schickt ein HTTP GET Request mithilfe des REST-Clients(?) an den Google-API-Server, welcher dann den Google-ID-Token zurückschickt. Der Token wird nun mithilfe der Methode sendGoogleIdToken() an den Server bzw. dem zugehörigen UserController gesendet. Im UserController wird die Methode registerOrLogin() aufgerufen, die (irgendwas mit JSON)...? Im UserService des Servers wird ebenfalls eine Methode registerOrLogin() aufgerufen. Das UserRepository stellt eine Brücke zwischen der Daten und der App dar und speichert nun also die User-Daten in der Datenbank ab. Auf dem Rückweg zur Activity wird schlussendlich der User übergeben und eine Response gesendet.