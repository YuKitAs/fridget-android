\documentclass[a4paper]{scrreprt}


%% Language and font encodings
\usepackage[german]{babel}
\setcounter{secnumdepth}{3} 
\setcounter{tocdepth}{3} 
\usepackage[utf8x]{inputenc}
\usepackage[T1]{fontenc}
\usepackage{courier}

%% Sets page size and margins
\usepackage[a4paper,top=3cm,bottom=2cm,left=3cm,right=3cm,marginparwidth=1.75cm]{geometry}

%% Useful packages
\usepackage{amsmath}
\usepackage{graphicx}
\usepackage[colorinlistoftodos]{todonotes}
\usepackage[colorlinks=true, allcolors=blue]{hyperref}

\title{Entwurfsdokument}
\author{   Yunjia Chen, Jasmin Jat, Min Hye Park, Alina Shah, Lisa Wang}

\begin{document}
\maketitle
\tableofcontents 
\newpage
\chapter{Einleitung}

\chapter{Grobentwurf}
	\section{Entwurfsentscheidungen}
    	\subsection{Activity}
        Unsere App setzt sich aus 15 Activites zusammen. Die Activities stellen die Benutzerschnittstelle unserer App dar und kümmern sich um die Interaktionen mit unserer Benutzeroberfläche. In jeder Activity-Klasse befinden sich selbstverständlich die üblichen Methoden eines Activity-Lifecycles: onCreate() zum Erstellen, onStart() zum Starten, onPause() zum Pausieren , onResume() zum Fortsetzen, onStop() zum Stoppen und onDestroy() zum Zerstören der Activity.
        
        
\newpage

\chapter{Feinentwurf}
	\section{Klassen des Clients}
    	\subsection{package kit.edu.pse.fridget.client.activity}
    	\subsubsection{\texttt{public class AppCompatActivity}}
               
               	\textbf{Beschreibung} \\
      	        \textit{Basisklasse für alle Activities} \\
                
                \textbf{Methoden}
                \begin{itemize}
        		\item{\texttt{public void onCreate(@Nullable Bundle savedInstanceState)}}\\
                \textit{Hier wird das Layout der Activity erstellt.}\\
                \end{itemize}
                
                \textbf{Parameter}
                \begin{itemize}
        		\item\texttt{Bundle savedInstanceState}\\ 
                \textit{Die zuvor gespeicherte Instanz der Activity, die wieder hergestellt zwird, sonst NULL}\\
                \end{itemize}
                
    	\subsubsection{\texttt{public static interface View.OnClickListener}}
        
      	        \textbf{Beschreibung} \\
               	\textit{Schnittstelle dafür, wenn auf die View geklickt wird.klickt wird.}\\

                \textbf{Methoden}
                \begin{itemize}
                \item{\texttt{public void onClick(View v)}}\\
                \textit{Aufruf bei einen Klick auf ein Element.}\\
                \end{itemize}

                \textbf{Parameter}
                \begin{itemize}
                \item\texttt{View v}\\
                \textit{Die angeklickte View}\\
                \end{itemize} 
                
        \subsubsection{\texttt{public static interface SwipeRefreshLayout.OnRefreshListener}}
        
      	        \textbf{Beschreibung} \\
     	        \textit{Schnittstelle dafür, wenn durch Hinunter-Swipen eine Aktualisierung ausgeführt werden soll} \\
                
                \textbf{Methoden}
                \begin{itemize}
        		\item\texttt{{public boolean onRefresh()}}\\
                \textit{Aufruf beim Hinunter-Swipen zum Aktualisieren}\\
                \end{itemize}       
                
      	\subsubsection{\texttt{public class LoginActivity \textbf{extends} AppCompatActivity \textbf{implements} View.OnClickListener}}
        
      	        \textbf{Beschreibung} \\
     	        \textit{Diese Klasse zeigt den Login mit dem Google-Account. Man kann seinen Google-Account-Daten eingeben und sich anmelden.} \\
                
                \textbf{Methoden}
                \begin{itemize}
        		\item\texttt{{public void onCreate(@Nullable Bundle savedInstanceState)}}\\
                \textit{Hier wird das Layout der Activity erstellt.}\\
                \end{itemize}
                
                \textbf{Parameter}
                \begin{itemize}
        		\item\texttt{Bundle savedInstanceState}\\ 
                \textit{Die zuvor gespeicherte Instanz der Activity, die wieder hergestellt zwird, sonst NULL}\\
                \end{itemize}
                
        \subsubsection{\texttt{public class StartActivity extends AppCompatActivity implements View.OnClickListener}}
               
               	\textbf{Beschreibung} \\
      	        \textit{Diese Klasse zeigt den Startbildschirm mit dem App-Logo und zwei Buttons: Ein Button zum Erstellen einer WG und einer zum Eingeben eines Zugangscodes.} \\
                
                \textbf{Methoden}
                \begin{itemize}
        		\item\texttt{{public void onCreate(@Nullable Bundle savedInstanceState)}}\\
                \textit{Hier wird das Layout der Activity erstellt.}\\
                \end{itemize}
                
                \textbf{Parameter}
                \begin{itemize}
        		\item\texttt{Bundle savedInstanceState}\\ 
                \textit{Die zuvor gespeicherte Instanz der Activity, die wieder hergestellt zwird, sonst NULL}\\
                \end{itemize}       
                
        \subsubsection{\texttt{public class CreateFlatshareActivity extends AppCompatActivity implements View.OnClickListener}}
               
               	\textbf{Beschreibung} \\
      	        \textit{In dieser Klasse kann man der WG einen Namen geben und kann mithilfe eines Buttons zur HomeActivity gelangen.} \\
                
                \textbf{Methoden}
                \begin{itemize}
        		\item\texttt{{public void onCreate(@Nullable Bundle savedInstanceState)}}\\
                \textit{Hier wird das Layout der Activity erstellt.}\\
                \end{itemize}
                
                \textbf{Parameter}
                \begin{itemize}
        		\item\texttt{Bundle savedInstanceState}\\ 
                \textit{Die zuvor gespeicherte Instanz der Activity, die wieder hergestellt zwird, sonst NULL}\\
                \end{itemize}      
                
        \subsubsection{\texttt{public class GetAccessCodeActivity extends AppCompatActivity implements View.OnClickListener}}
               
               	\textbf{Beschreibung} \\
      	        \textit{In dieser Klasse kriegt man den Zugangscode und kann mithilfe eines Buttons zur HomeActivity gelangen.} \\
                
                \textbf{Methoden}
                \begin{itemize}
        		\item\texttt{{public void onCreate(@Nullable Bundle savedInstanceState)}}\\
                \textit{Hier wird das Layout der Activity erstellt.}\\
                \end{itemize}
                
                \textbf{Parameter}
                \begin{itemize}
        		\item\texttt{Bundle savedInstanceState}\\  
                \textit{Die zuvor gespeicherte Instanz der Activity, die wieder hergestellt zwird, sonst NULL}\\
                \end{itemize}  
                
        \subsubsection{\texttt{public class EnterAccessCodeActivity extends AppCompatActivity implements View.OnClickListener}}
               
               	\textbf{Beschreibung} \\
      	        \textit{In dieser Klasse kann man den Zugangscode eingeben und kann mithilfe eines Buttons zur HomeActivity gelangen.} \\
                
                \textbf{Methoden}
                \begin{itemize}
        		\item\texttt{{public void onCreate(@Nullable Bundle savedInstanceState)}}\\
                \textit{Hier wird das Layout der Activity erstellt.}\\
                \end{itemize}
                
                \textbf{Parameter}
                \begin{itemize}
        		\item\texttt{Bundle savedInstanceState}\\ 
                \textit{Die zuvor gespeicherte Instanz der Activity, die wieder hergestellt zwird, sonst NULL}\\
                \end{itemize} 
                
        \subsubsection{\texttt{public class HomeActivity extends AppCompatActivity implements View.OnClickListener implements SwipeRefreshLayout.OnRefreshListener}}
               
               	\textbf{Beschreibung} \\
      	        \textit{Diese Klasse zeigt das View der WG-Pinnwand, man sieht die Notes mit Überschrift und Magnet und einige Buttons. Drei Frozen Notes sind von Anfang an enthalten. Frozen Notes haben immer einen schwarzen Magneten.} \\
                
                \textbf{Methoden}
                \begin{itemize}
        		\item\texttt{{public void onCreate(@Nullable Bundle savedInstanceState)}}\\
                \textit{Hier wird das Layout der Activity erstellt.}\\
                \end{itemize}
                
                \textbf{Parameter}
                \begin{itemize}
        		\item\texttt{Bundle savedInstanceState}\\  
                \textit{Die zuvor gespeicherte Instanz der Activity, die wieder hergestellt zwird, sonst NULL}\\
                \end{itemize} 
                
       	\subsubsection{\texttt{public class FullTextCoolNoteActivity extends AppCompatActivity implements View.OnClickListener implements SwipeRefreshLayout.OnRefreshListener}}
               
               	\textbf{Beschreibung} \\
      	        \textit{Diese Klasse zeigt eine Großansicht einer Text-Cool-Note mit zugehörigem Magneten, Erstelldatum, Tags, Titel, Inhalt, Lesebestätigungen und Kommentaren. Der @All-Tag ist immer da, wenn keine Tags spezifiziert werden. Es stehen wieder einige Buttons zur Interaktion zu Verfügung.} \\
                
                \textbf{Methoden}
                \begin{itemize}
        		\item\texttt{{public void onCreate(@Nullable Bundle savedInstanceState)}}\\
                \textit{Hier wird das Layout der Activity erstellt.}\\
                \end{itemize}
                
                \textbf{Parameter}
                \begin{itemize}
        		\item\texttt{Bundle savedInstanceState}\\ 
                \textit{Die zuvor gespeicherte Instanz der Activity, die wieder hergestellt zwird, sonst NULL}\\
                \end{itemize} 
                
      	\subsubsection{\texttt{public class FullFrozenNoteActivity extends AppCompatActivity implements View.OnClickListener implements SwipeRefreshLayout.OnRefreshListener}}
               
               	\textbf{Beschreibung} \\
      	        \textit{Diese Klasse zeigt eine Großansicht einer Frozen Note mit zugehörigem schwarzen Magneten, Titel und Inhalt. Es stehen wieder einige Buttons zur Interaktion zu Verfügung.} \\
                
                \textbf{Methoden}
                \begin{itemize}
        		\item\texttt{{public void onCreate(@Nullable Bundle savedInstanceState)}}\\
                \textit{Hier wird das Layout der Activity erstellt.}\\
                \end{itemize}
                
                \textbf{Parameter}
                \begin{itemize}
        		\item\texttt{Bundle savedInstanceState}\\ 
                \textit{Die zuvor gespeicherte Instanz der Activity, die wieder hergestellt zwird, sonst NULL}\\
                \end{itemize} 
        
        \subsubsection{\texttt{public class FullImageCoolNoteActivity extends AppCompatActivity implements View.OnClickListener implements SwipeRefreshLayout.OnRefreshListener}}
               
               	\textbf{Beschreibung} \\
      	        \textit{Diese Klasse zeigt eine Großansicht einer Bild-Cool-Note mit zugehörigem Magneten, Erstelldatum, Tags, Titel, Inhalt, Lesebestätigungen und Kommentaren. Der @All-Tag ist immer da, wenn keine Tags spezifiziert werden. Es stehen wieder einige Buttons zur Interaktion zu Verfügung.} \\
                
                \textbf{Methoden}
                \begin{itemize}
        		\item\texttt{{public void onCreate(@Nullable Bundle savedInstanceState)}}\\
                \textit{Hier wird das Layout der Activity erstellt.}\\
                \end{itemize}
                
                \textbf{Parameter}
                \begin{itemize}
        		\item\texttt{Bundle savedInstanceState}\\ 
                \textit{Die zuvor gespeicherte Instanz der Activity, die wieder hergestellt zwird, sonst NULL}\\
                \end{itemize} 
                
        \subsubsection{\texttt{public class CreateTextCoolNoteActivity extends AppCompatActivity implements View.OnClickListener}}
               
               	\textbf{Beschreibung} \\
      	        \textit{Diese Klasse zeigt das View für die Erstellung einer Text-Cool-Note. Dieses View wird auch für die Kommentar-Funktion benutzt, nur, dass man keinen Titel schreiben und keine Wichtigkeit auswählen kann. Das View öffnet sich auch, wenn man eine Frozen Note editieren will, wobei die Wichtigkeit wieder nicht auswählbar ist.} \\
                
                \textbf{Methoden}
                \begin{itemize}
        		\item\texttt{{public void onCreate(@Nullable Bundle savedInstanceState)}}\\
                \textit{Hier wird das Layout der Activity erstellt.}\\
                \end{itemize}
                
                \textbf{Parameter}
                \begin{itemize}
        		\item\texttt{Bundle savedInstanceState}\\  
                \textit{Die zuvor gespeicherte Instanz der Activity, die wieder hergestellt zwird, sonst NULL}\\
                \end{itemize} 
          
         
     	\subsubsection{\texttt{public class CreateImageCoolNoteActivity extends AppCompatActivity implements View.OnClickListener}}
               
               	\textbf{Beschreibung} \\
      	        \textit{Diese Klasse zeigt das View für die Erstellung einer Bild-Cool-Note.} \\
                
                \textbf{Methoden}
                \begin{itemize}
        		\item\texttt{{public void onCreate(@Nullable Bundle savedInstanceState)}}\\
                \textit{Hier wird das Layout der Activity erstellt.}\\
                \end{itemize}
                
                \textbf{Parameter}
                \begin{itemize}
        		\item\texttt{Bundle savedInstanceState}\\ 
                \textit{Die zuvor gespeicherte Instanz der Activity, die wieder hergestellt zwird, sonst NULL}\\
                \end{itemize} 
        
        \subsubsection{\texttt{public class CreateCommentActivity extends AppCompatActivity implements View.OnClickListener}}
               
               	\textbf{Beschreibung} \\
      	        \textit{Diese Klasse zeigt das View für die Erstellung eines Kommentars.} \\
                
                \textbf{Methoden}
                \begin{itemize}
        		\item\texttt{{public void onCreate(@Nullable Bundle savedInstanceState)}}\\
                \textit{Hier wird das Layout der Activity erstellt.}\\
                \end{itemize}
                
                \textbf{Parameter}
                \begin{itemize}
        		\item\texttt{Bundle savedInstanceState}\\  
                \textit{Die zuvor gespeicherte Instanz der Activity, die wieder hergestellt zwird, sonst NULL}\\
                \end{itemize} 
                
                
        \subsubsection{\texttt{public class EditFrozenNoteActivity extends AppCompatActivity implements View.OnClickListener}}
               
               	\textbf{Beschreibung} \\
      	        \textit{Diese Klasse zeigt das View für die Bearbeitung einer Frozen Note.} \\
                
                \textbf{Methoden}
                \begin{itemize}
        		\item\texttt{{public void onCreate(@Nullable Bundle savedInstanceState)}}\\
                \textit{Hier wird das Layout der Activity erstellt.}\\
                \end{itemize}
                
                \textbf{Parameter}
                \begin{itemize}
        		\item\texttt{Bundle savedInstanceState}\\ 
                \textit{Die zuvor gespeicherte Instanz der Activity, die wieder hergestellt zwird, sonst NULL}\\
                \end{itemize} 
                
        \subsubsection{\texttt{public class MemberListActivity extends AppCompatActivity implements View.OnClickListener implements SwipeRefreshLayout.OnRefreshListener}}
               
               	\textbf{Beschreibung} \\
      	        \textit{Diese Klasse zeigt das View zum Einsehen der aktuellen Mitglieder mit
 mit zugehörigem Magneten.} \\
                
                \textbf{Methoden}
                \begin{itemize}
        		\item\texttt{{public void onCreate(@Nullable Bundle savedInstanceState)}}\\
                \textit{Hier wird das Layout der Activity erstellt.}\\
                \end{itemize}
                
                \textbf{Parameter}
                \begin{itemize}
        		\item\texttt{Bundle savedInstanceState}\\  
                \textit{Die zuvor gespeicherte Instanz der Activity, die wieder hergestellt zwird, sonst NULL}\\
                \end{itemize} 
               
       \subsubsection{\texttt{public class LanguageActivity extends AppCompatActivity implements View.OnClickListener}}
               
               	\textbf{Beschreibung} \\
      	        \textit{Diese Klasse zeigt das View zum Einstellen der App-Sprache.} \\
                
                \textbf{Methoden}
                \begin{itemize}
        		\item\texttt{{public void onCreate(@Nullable Bundle savedInstanceState)}}\\
                \textit{Hier wird das Layout der Activity erstellt.}\\
                \end{itemize}
                
                \textbf{Parameter}
                \begin{itemize}
        		\item\texttt{Bundle savedInstanceState}\\  
                \textit{Die zuvor gespeicherte Instanz der Activity, die wieder hergestellt zwird, sonst NULL}\\
                \end{itemize} 
        
        \subsubsection{\texttt{public abstract class ResultReceiver}}
               
               	\textbf{Beschreibung} \\
      	        \textit{Ist ein generisches Interface um callback results zu erhalten.} \\
               
             	\textbf{Konstuktoren}
                \begin{itemize}
        		\item\texttt{{public ResultReceiver(Handler handler)}}\\
                \end{itemize}
               
                \textbf{Methoden}
                \begin{itemize}
        		\item\texttt{{public void onReceiverResult(int resultCode, Bundle resultData)}}\\
                \textit{Zum Bekommen und Bearbeiten der Ergebnisse}\\
                \end{itemize}
                
                \textbf{Parameter}
                \begin{itemize}
        		\item\texttt{Handler handler}\\ 
                \textit{Zum Bearbeiten und Schicken von Nachrichten und Runnable-Objekten}\\
				\item\texttt{Integer resultCode}\\ 
                \textit{Vom Sender definierter Ergebnis-Code}\\
                \item\texttt{Bundle resultData}\\ 
                \textit{Weitere Daten vom Sender }\\
                \end{itemize}
                
        \subsubsection{\texttt{public class ServiceResultReceiver extends ResultReceiver implements Receiver}}
               
               	\textbf{Beschreibung} \\
      	        \textit{Erweiterung der Klasse} \texttt{ResultReceiver}\textit{, implementiert das Interface} \texttt{Receiver}\\
                
                \textbf{Methoden}
                \begin{itemize}
        		\item\texttt{{public void setReceiver(Receiver receiver)}}\\
                \textit{Zum Setzen des Receivers}\\
                \item\texttt{{public void onReceiverResult(int resultCode, Bundle resultData)}}\\
                \textit{Zum Bekommen und Bearbeiten der Ergebnisse}\\
                \end{itemize}
                
                \textbf{Parameter}
                \begin{itemize}
        		\item\texttt{Receiver receiver}\\ 
                \textit{Broadcast Receiver}\\
				\item\texttt{Integer resultCode}\\ 
                \textit{Vom Sender definierter Ergebnis-Code}\\
                \item\texttt{Bundle resultData}\\ 
                \textit{Weitere Daten vom Sender }\\
                \end{itemize}
        
        \newpage
                
                
		\subsection{package kit.edu.pse.fridget.client.service}
		\subsubsection{\texttt{public interface UserService}}
        \textit{Dieses Interface dient dazu, dem Nutzer zu ermöglichen sein bestehendes Google Account zum Login zu verwenden. Dabei sendet der Client beim Login ein Token, welches er vom GoogleServer erhält. Dort wird der Token an den GoogleServer gesendet, dadurch erhält der Server die Google Client Id. Diese wird dauerhaft gespeichert.}\\
        
		\textbf{Methoden} \\
 			\begin{itemize}
        		\item{@GET\\ Call<User> getGoogleIdToken(String googleIdToken)}
        	
        		\textit{Diese Methode ruft den GoogleToken vom Goolge API Server ab}
        	
        		\textbf{Parameter} \\
                googleIDToken - zum speichern des GoogleToken
        		        	
       		 	\textbf{Rückgabewert} \\
                GoogleToken
      		  	 
      	      \item{@POST\\ Call<User> sendGoogleIdToken(String googleIdToken)}
        	
      	 	 	\textit{Diese Methode sendet den GoogleIdToken an den Server}
        	
        		\textbf{Parameter} \\
        		googleIdToken - zu sendender GoogleToken
        	
        		\textbf{Rückgabewert} \\
                GoogleToken
        	
       		 \end{itemize}
             
             	\subsubsection{\texttt{public interface AccesCodeService}}
        \textit{Dieses Interface ist für die Synchronisation des Access-Codes mit dem Server zuständig }\\
        \\
		\textbf{Methoden} \\
 			\begin{itemize}
        		\item{@Headers()\\ Call<AccessCode> getAccessCode(String flatShareID)}
        	
        		\textit{Diese Methode Diese Methode fordert den Accesscode einer Flatshare an}
        	
        		\textbf{Parameter} \\
                flatshareId - übergebende ID der Flatshare
        		        	
       		 	\textbf{Rückgabewert} \\
                AccessCode der flatshare
      		  	 
      	        	
       		 \end{itemize}
             
             		\subsubsection{\texttt{public interface CommentService }}
        \textit{Dieses interfache ist für die Synchronisation der Comments mit dem Server zuständig}\\
        \\
		\textbf{Methoden} \\
 			\begin{itemize}
        		\item{@Headers()\\@GET("comments?cool-note=\{cid\}")\\ Call<List<Comment>> getAllComments(@Path(\grqq cid\grqq)String coolNoteId)}
        	
        		\textit{Diese Methode ruft alle Kommentare einer CoolNote ab}
        	
        		\textbf{Parameter} \\
                 coolNoteId - die Id der CoolNote 
        		        	
       		 	\textbf{Rückgabewert} \\
                Alle Comments der zur CoolNoteId zugehörigen CoolNote
                
      		  	 
      	      \item{@Headers()\\@POST("/comments")
\\ Call<Comment> createComment(@Body Comment comment)}
        	
      	 	 	\textit{Diese Methode schickt ein Comment an den Server }
        	
        		\textbf{Parameter} \\
        		 comment - speichert ein Comment
        	
        		\textbf{Rückgabewert} \\
                Ein Comment
                
                 \item{@Headers()\\@DELETE("/comments/{id}")\\ Call<Comment> deleteComment (@Path(\grqq id\grqq)String commentID)}
        	
      	 	 	\textit{Diese Methode löscht ein Comment }
        	
        		\textbf{Parameter} \\
        		 commentID - ID des zu löschenden Comments 
        	
        	                             
        	
       		 \end{itemize}
             
             
             	\subsubsection{\texttt{public interface CoolNoteService }}
        \textit{Dieses Interface ist für die Synchronisation der CoolNotes mit dem Server zuständig}\\
        \\
		\textbf{Methoden} \\
 			\begin{itemize}
        		\item{@Headers()\\@GET("/cool-notes?flatshare={id}")\\ Call<List<CoolNote>> getAllCoolNotes(@Path(\grqq id\grqq) String flatshareID)} 
        	
        		\textit{Diese Methode ruft alle CoolNotes einer flatshare ab}
        	
        		\textbf{Parameter} \\
                flatshareId - Die flatshare der abzurufenden CoolNotes
        		        	
       		 	\textbf{Rückgabewert} \\
                Alle Cool Notes
      		  	 
      	      \item{@Headers() \\ @GET("/cool-notes/{id}")\\ Call<CoolNote> getCoolNote(@Path(\grqq id\grqq) String coolNoteId)}
        	
      	 	 	\textit{Diese Methode ruft den Inhalt einer CoolNote ab }
        	
        		\textbf{Parameter} \\
        		coolNoteId - Die Id der CoolNote 
        	
        		\textbf{Rückgabewert} \\
                Inhalt der zu der CoolNoteId gehörenden CoolNote
        	
              \item{@Headers() \\ @POST("/cool-notes")\\ Call<CoolNote> createCoolNote(@Body CoolNote coolNote)}
        	
      	 	 	\textit{Diese Methode schickt eine neue CoolNote an den Server }
        	
        		\textbf{Parameter} \\
        		coolNoteI - Die CoolNote 
        	
        		\textbf{Rückgabewert} \\
                CoolNote
        	
            
              \item{@Headers() \\ @DELETE("/cool-notes/{id}")\\ Call<CoolNote> deleteCoolNote(@Path(\grqq id\grqq) String coolNoteId)}
        	
      	 	 	\textit{Diese Methode löscht eine CoolNote }
        	
        		\textbf{Parameter} \\
        		coolNoteId - Die Id der CoolNote 
        	
                                
       		 \end{itemize}
             
             
             	\subsubsection{\texttt{public interface  FlatShareService }}
        \textit{Dieses Interface verwaltet die Synchronisation der Flatshare mit dem Server}\\
        \\
		\textbf{Methoden} \\
 			\begin{itemize}
        		\item{@Headers()\\ @POST ("/flatshares") \\
   Call<Flatshare> createFlatshare(@Body Flatshare flatshare)
}
        	
        		\textit{Diese Methode erstellt eine neue Flatshare auf dem Server
}
        	
        		\textbf{Parameter} \\
                flatshare - zu erstellende Flatshare 
        		        	
       		 	\textbf{Rückgabewert} \\
                Flatshare
      		  	 
      	      \item{@Headers()\\  @GET("/flatshares/{id}")\\ Call<Flatshare> getFlatshare(@Path("id") 					String flatshareId)}
        	
      	 	 	\textit{Diese Methode ruft eine Flatshare Daten vom Server ab }
        	
        		\textbf{Parameter} \\
        		flatshareId - die ID der aufgerufenen Flatshare 
        	
        		\textbf{Rückgabewert} \\
                Flatshare
                      	
       	
       		 \end{itemize}
             
             
             
             	\subsubsection{\texttt{public interface FrozenNoteService }}
        \textit{Dieses Interface ist für die synchronisation der FrozenNotes mit dem Server zuständig}\\
        \\
		\textbf{Methoden} \\
 			\begin{itemize}
        		\item{@Headers\\ @GET("/frozen-notes?flatshare={id}")\\
   Call<List<FrozenNote>> getAllFrozenNote(@Path("id") String flatShareId)}
        	
        		\textit{Diese Methode ruft die FrozenNotes vom Server ab}
        	
        		\textbf{Parameter} \\
                flatshareId -  die ID der aufgerufenen Flatshare  
        		        	
       		 	\textbf{Rückgabewert} \\
                FrozenNote
      		  	 
      	      \item{@Headers\\ @GET("/frozen-notes/{id}")\\ Call<FrozenNote> getFrozenNote(@Path("id") String frozenNoteId)}
        	
      	 	 	\textit{Diese Methode ruft den Inhalt einer FrozenNote ab }
        	
        		\textbf{Parameter} \\
        		frozenNoteId - die ID der aufgerufenen FrozenNote  
        	
        		\textbf{Rückgabewert} \\
                FrozenNote
                
                 \item{@Headers\\ @PUT("/frozen-notes/{id}")\\ Call<FrozenNote> updateFrozenNote(@Path("id") String frozenNoteId, @Body FrozenNote frozenNote)}
        	
      	 	 	\textit{Diese Methode speichert Änderungen  in einer FrozenNote}
        	
        		\textbf{Parameter} \\
        		frozenNoteId - die ID der aufgerufenen FrozenNote  
        		frozenNote - die geänderte FrozenNote
        		\textbf{Rückgabewert} \\
                FrozenNote
        	
       		 \end{itemize}
             
             
             	\subsubsection{\texttt{public interface ImageNoteService }}
        \textit{Dieses Interface dient zur Synchronisation der ImageCoolNotes mit dem Server}\\
        \\
		\textbf{Methoden} \\
 			\begin{itemize}
        		\item{@Headers()\\ @GET("/image-notes?flatshare={id}")\\
   Call<List<ImageNote>> getAllImageNotes(@Path("id") String flatshareId)}
        	
        		\textit{Diese Methode ruft die ImageCoolNotes vom server ab}
        	
        		\textbf{Parameter} \\
                flatshareId - die ID der aufgerufenen Flatshare   
        		        	
       		 	\textbf{Rückgabewert} \\
                ImageCoolNote
      		  	 
      	      \item{@Heders\\ @GET("/image-notes/{id}")\\ Call<ImageNote> getImageNote(@Path("id") String imageNoteId)}
        	
      	 	 	\textit{Diese Methode ruft eine ImageNote ab }
        	
        		\textbf{Parameter} \\
        		 imageNoteId - die ID der aufgerufenen ImageNote  
        	
        		\textbf{Rückgabewert} \\
                ImageNote
                
                \item{@Headers\\ @POST("/image-notes")\\ Call<ImageNote> createImageNote( @Body ImageNote imageNote)}
        	
      	 	 	\textit{Diese Methode schickt ein ImageNote an den Server}
        	
        		\textbf{Parameter} \\
        		 imageNote - eine ImageNote  
        	
        		\textbf{Rückgabewert} \\
                ImageNote
                
                     \item{@Headers\\ @DELETE("/image-notes/{id}")\\Call<ImageNote> deleteImageNote(@Path("id") String ImageNoteId)}
        	
      	 	 	\textit{Diese Methode löscht eine ImageNote}
        	
        		\textbf{Parameter} \\
        		 imageNoteId - Die ID der zu löschenden ImageNote  
        	        		       	
       		 \end{itemize}
             
             
             	\subsubsection{\texttt{public interface  MembershipService }}
        \textit{Dieses Interface verwaltet die Synchronisation der Members mit dem Server}\\
        \\
		\textbf{Methoden} \\
 			\begin{itemize}
        		\item{@Headers()\\ @GET("/memberships/users?flatshare={id}") \\ Call<List<Membership>> getMemberList(@Path("flatshareID") String flatshareID)}
        	
        		\textit{Diese Methode ruft die Mitglieder einer Flatshare ab}
        	
        		\textbf{Parameter} \\
                flatshareId - die ID der aufgerufenen Flatshare  
        		        	
       		 	\textbf{Rückgabewert} \\
                MemberList
      		  	 
      	      \item{@Headers()\\ @GET("memberships?flatshare={fid}\&user ={uid}")\\Call<Membership> getUser(@Path("fid") String flatshareId, @Path("uid") String userid)}
        	
      	 	 	\textit{Diese Methode ruft die Daten eines Members ab}        	
        		\textbf{Parameter} \\
        		flatshareId - die ID der aufgerufenen Flatshare 
                userId - die ID des aufgerufenen Users
        	
        		\textbf{Rückgabewert} \\
              Daten eines Members
        	
                      	 
      	      \item{@Headers()\\ @POST("/memberships")\\ Call<Membership> createMembership(@Body User user)}
        	
      	 	 	\textit{Diese Methode fügt ein neues Member in eine flatshare ein}        	
        		\textbf{Parameter} \\
        		user - Der User der zu der flatshare hinzugefügt wird 
        	
             	      \item{@Headers()\\ @DELETE("/memberships?flatshare={fid}\&user={uid}")\\Call<Membership> deleteMember(@Path("fid") String flatshareId, @Path("uid") String userId)}
        	
      	 	 	\textit{Diese Methode löscht ein Member}        	
        		\textbf{Parameter} \\
        		flatshareId - die ID der aufgerufenen Flatshare 
                userId - die ID des aufgerufenen Users
        	       		             
        	             
       		 \end{itemize}
             
             	\subsubsection{\texttt{public interface  ReadConfirmationService }}
        \textit{ Dieses Interface synchronisasiert den gelesen Status mit dem Server}\\
        \\
		\textbf{Methoden} \\
 			
            \begin{itemize}
        		\item{@Headers()\\@GET("/read-confirmations/users?cool-note={id}") \\ Call<List<User>> getReadStatus(@Path("id") String coolNoteId)}
        	
        		\textit{Diese Methode ruft den gelesen Status vom Server ab}
        	
        		\textbf{Parameter} \\
                 coolNoteId - Die ID der betreffenden CoolNote
        		        	
       		 	\textbf{Rückgabewert} \\
                ReadStatus
      		  	 
      	     	\item{@Headers()\\ @POST("/read-confirmations") \\ Call<CoolNote> createReadStatus(@Body Readstatus readstatus) } \todo{Mins Klasse übernehmen}
        	
        		\textit{Diese Methode setzt die Checkbox auf markiert}
        	
        		\textbf{Parameter} \\
                 readstatus - zeigt den gelesen Status einer CoolNote an
        		        	
       		 	\textbf{Rückgabewert} \\
                ReadStatus
        	
       		 
            	  	 
      	     	\item{@Headers()\\ @DELETE("/read-confirmations?cool-note={cid}\&user={uid}")}
 \\Call<CoolNote> deleteReadStatus(@Path("cid") String coolNoteId,
                                     @Path("uid") String userID);
        	
        		\textit{Diese Methode setzt die Checkbox auf unmarkiert}
        	
        		\textbf{Parameter} \\
                coolNoteID - die ID der aufgerufenen CoolNote
                userID - die ID des aufgerufenen Users
        		        	
       		 \todo{Rückgabewert bei DELETE und PUT}
              
        	
       		 \end{itemize}
            
                
             	\subsubsection{\texttt{public interface DeviceService }}
        \textit{Dieses Interface synchronisiert die Device-Daten mit dem Server}\\
        \\
		\textbf{Methoden} \\
 			\begin{itemize}
        		\item{@Headers()\\@POST("/devices") \\ createDevice Call<@Body User user> ()}
        	
        		\textit{Diese Methode fügt ein Device zu einer flatshare hinzu}
        	
        		\textbf{Parameter} \\
                 user - der zu dem Device zugehörige User 
        		        	
       		 	\textbf{Rückgabewert} \\
                
      		  	  	     	
                
        	
       		 \end{itemize}
            
             
	    
             	\subsubsection{\texttt{public interface }}
        \textit{}\\
        \\
		\textbf{Methoden} \\
 			\begin{itemize}
        		\item{@\\ \\ Call<> ()}
        	
        		\textit{Diese Methode}
        	
        		\textbf{Parameter} \\
                 - 
        		        	
       		 	\textbf{Rückgabewert} \\
                
      		  	 
      	     	\item{@\\ \\ Call<> ()}
        	
        		\textit{Diese Methode}
        	
        		\textbf{Parameter} \\
                 - 
        		        	
       		 	\textbf{Rückgabewert} \\
                
        	
       		 \end{itemize}

\section{Klassen des Servers}
 



\chapter{Datenstrukturen}

\chapter{Dynamische Modelle}

\chapter{Nicht entworfene Wunschkriterien}

\chapter{Glossar}

\chapter{Anhang}



\end{document}