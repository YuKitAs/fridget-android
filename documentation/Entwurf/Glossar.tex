\begin{table}[h!]
			\centering
			\label{my-label}
			\begin{tabular}{p{4cm}p{10cm}}
				\textbf{MVVM} & Model View ViewModel  \\
				\textbf{MVC} & Model View Controller  \\
				\textbf{Data Binding} & Automatische Datenweitergabe zwischen Objekten  \\
				\textbf{Client} & User der die App Fridget benutzt   \\
				
				\textbf{Server} & a    \\
				\textbf{Activity} & Stellt eine Bildschimrseite in einer App dar   \\
				\textbf{Geschäftslogik} & Mittelschicht einer mehrschichitgen Anwendung  \\
				\textbf{Activity-Lifecycle}& Stufen die eine Activity während ihres Lebens durchschreitet   \\
				
				\textbf{API} & Application Programming Interface - Schnittstelle die ein Softwaresystem bereitstellt um dieses in andere Programme einzubinden  \\
				
			
				\textbf{HTTP} & Zustandsloses Protokoll zur Übertragung von Daten auf der Anwendungsschicht über ein Rechnernetz  \\
				\textbf{URL} & Uniform Resource Locator  \\
				\textbf{Open-Source} & Quelltext, welcher öffentlich und von dritten eingesehen, geändert und genutzt werden kann  \\
				
				\textbf{MR-Remote API} & a  \\
				\textbf{JSON} & JavaScript Object Notation - Datenaustauschformat, das für Menschen einfach zu lesen und für Maschinen einfach zu analysieren und generieren ist   \\
				\textbf{POJO} & Plain Old Java Object  -   \\
				\textbf{Framework} & a  \\
				
				\textbf{Instanz} & a  \\
				\textbf{Tag / Taggen} &  Markierung und namentliche Erwähnung von Mitgliedern auf Notizzetteln
   \\
				\textbf{Cool-Note} & Notizen, die löschbar, nicht bearbeitbar sind und vom Benutzer erstellt werden   \\
				\textbf{Layout} & a  \\
				
				\textbf{Button} & a\\
				\textbf{FrozenNote} & Notizen, die fest, nicht löschbar, bearbeitbar sind und beim Erstellen der WG generiert werden.   \\
				\textbf{Nullable} & a  \\
				\textbf{NULL} & a  \\
				\textbf{Local Repository} & a  \\
				\textbf{Parameter} & a  \\
				\textbf{Methoden} & a  \\
				\textbf{Tutorial} &  Einführung in die Funktionen der App
   \\
				\textbf{Check Box} &  Eine Box, die abgehakt werden kann \\
				\textbf{Interface} & a  \\
				\textbf{Token} & a  \\
				\textbf{Headers} & a  \\
				\textbf{Endpoints} & a  \\
				\textbf{ResponseEntity} & a  \\	
				\textbf{Getter} & a  \\	
				\textbf{UUID} &a   \\	
				\textbf{ID} &  a \\	
				\textbf{Builder} &  a \\	
				\textbf{JWT} &  a \\	
				\textbf{Repository} &  a \\	
				
			\end{tabular}
		\end{table}