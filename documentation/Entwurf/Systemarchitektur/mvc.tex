Auf der Server-Seite wird das MVC (Model-View-Controller) Architektur-Muster eingesetzt. Dabei ist in unserem Falle die Client-Seite die View. Model ist für die Datenhaltung zuständig und der Controller steuert zwischen View, Model sowie Repository.  

Sobald auf Client-Seite eine Interaktion mit dem Benutzer und der Applikation geschieht, wird dies vom Service der Client-Seite weiter an dem Controller geleitet. Dieser kann die Veränderungen dann durchführen lassen.

