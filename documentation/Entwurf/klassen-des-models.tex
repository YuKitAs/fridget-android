\subsection{package kit.edu.pse.fridget.client.datamodel}
\subsubsection{\texttt{public class Accesscode}}

	\textbf{Beschreibung} \\
	\textit{Die Klasse Accesscode stellt der Access-Code, oder Zugangscode einer WG dar.} \\

	\textbf{Methoden}
	\begin{itemize}
		\item{\texttt{public String getAccessCodeID()}}\\
		\textit{Gibt die ID des Access-Codes der WG zurück.}\\
		\item{\texttt{public String getAccessCodeContent()}}\\
		\textit{Gibt den Access-Code als String zurück.}\\
	\end{itemize}

	

\subsubsection{\texttt{public class User}}

	\textbf{Beschreibung} \\
	\textit{Die Klasse User stellt einen Benutzer der App dar.}\\

	\textbf{Methoden}
	\begin{itemize}
	\item{\texttt{public String getUserID()}}\\
	\textit{Gibt die ID des Benutzers der App zurück.}\\
	\item{\texttt{public String getUserID()}}\\
	\textit{Gibt den Namen des Benutzers zurück.}\\
	\end{itemize}

	

\subsubsection{\texttt{public class Member}}

	\textbf{Beschreibung} \\
	\textit{Die Klasse Member stellt einen Mitglieder einer WG dar.} \\

	\textbf{Methoden}
	\begin{itemize}
		\item\texttt{{public String getMemberID()}}\\
		\textit{Gibt die ID des Benutzers der App zurück.}\\
		\item\texttt{{public Member getMember()}}\\
		\textit{Gibt den Mitglieder zurück.}\\
		\item\texttt{{public String getMagnet()}}\\
		\textit{Gibt den/die Magnet/Magnetfarbe des Benutzers zurück.}\\
	\end{itemize}       

\subsubsection{\texttt{public class Flatshare}}

	\textbf{Beschreibung} \\
	\textit{Die Klasse Flatshare stellt eine WG dar, die in das App registriert ist.} \\

	\textbf{Methoden}
	\begin{itemize}
		\item\texttt{{public String getFlatshareID()}}\\
		\textit{Gibt die ID der WG zurück.}\\
		\item\texttt{{public String getFlatshareName()}}\\
		\textit{Gibt den WG-Namen zurück.}\\
	\end{itemize}

\subsubsection{\texttt{public class CoolNote}}

	\textbf{Beschreibung} \\
	\textit{Die Klasse CoolNote stellt die Notiz der Art “Cool Note” dar, die aus einer Überschrift und schriftlichem Inhalt besteht. Cool Notes sind erstellbare, nicht editierbare, löschbare Notizen.} \\

	\textbf{Methoden}
	\begin{itemize}
		\item\texttt{{public String getCoolNoteID(CoolNote coolNoteID)}}\\
		\textit{Gibt die Cool-Note-ID zurück.}\\
		\textbf{Parameter}\\
		“CoolNote coolNoteID”: Die ID oder Kennnummer der Cool Note.\\
		\item\texttt{{public String getTitleCoolNote()}}\\
		\textit{Gibt die Überschrift einer Cool Note zurück.}\\
		\item\texttt{{public String getWriterID(Member memberID)}}\\
		\textit{Gibt die ID des Mitglieders zurück, der die Cool Note erstellt hat.}\\
		\textbf{Parameter}\\
		“UserID userID”: Die ID oder Kennnummer des Mitglieders, der die Cool Note erstellt hat.\\
		
		\item\texttt{{public String getContentCoolNote()}}\\
		\textit{Gibt den Inhalt der Cool Note zurück.}\\
		\item\texttt{{public List<Member> getTaggedMembers(Member member)}}\\
		\textit{Gibt die Mitglieder zurück, die in der Cool Note getaggt sind. Wenn keine spezifiziert ist, sind alle Mitglieder der WG getaggt.}\\
		\textbf{Parameter}\\
		“Member member”: Der Benutzer, der in der Cool Note getaggt wird.\\
		\item\texttt{{enum importance{NORMAL, IMPORTANT, IMMEDIATE}}}\\
		\item\texttt{{public int getPosition()}}\\
		\textit{Gibt die Position der Cool Note zurück.}\\
	\end{itemize}

