\documentclass[a4paper]{scrreprt}

\usepackage[german]{babel}
\usepackage[utf8]{inputenc}
\usepackage[T1]{fontenc}
\usepackage{ae}

\begin{document}
    \chapter{Produkteinsatz}
        \section{Anwendungsbereiche}
        Die App ist in einem gemeinsamen Haushalt einsetzbar, um wichtige interne Informationen unmittelbar unter allen Mitbewohnern auszutauschen.
        
        \section{Produktumgebung}
        \begin{tabular}{|l|l|p{.6\textwidth}|}
        \hline
        \multicolumn{3}{|l|} {Server} \\
        \hline
        Container & Docker & Die Serverseitigen Softwares werden in einem Docker Container verpackt. \\ \hline
        Webserver & Tomcat & Zum Übertragen von Daten an Client durch HTTP-Anfrage. \\ \hline
        Datenbank & MySQL & Zum Speichern und Verwalten von Daten. \\
        \hline \hline
        \multicolumn{3}{|l|}{Client} \\
        \hline
        Mobiles Betriebssystem & \multicolumn{2}{|l|}{Android 5.1 Lollipop} \\ \hline
        \end{tabular}
        
        \section{Betriebsbedingungen}
        Zur Anmeldung wird ein bereits vorhandenes Google-Konto benötigt. \\
        Zum App-Betrieb wird eine aktive Internetverbindung zum Server benötigt. Ohne eine bestehende Verbindung können weder die Notizen synchronisiert noch Benachrichtigungen erhalten werden.

        \section{Zielgruppen}
        Haupt-Zielgruppe der App sind die Personen, die in einer Wohngemeinschaft oder Haushaltsgemeinschaft leben. Ebenfalls kann sie von kleinen Gruppen genutzt werden, die nicht zusammen wohnen, sich jedoch trotzdem verbinden möchten.

\end{document}
